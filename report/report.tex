\documentclass{article}

\begin{document}

\title{Report Outline}
\author{Fabian}

\maketitle

\section{Introduction}
A short introduction to the field of load forecasting and its relevance.\\
Description of the competition.

\section{Review of related Work}
Review previous work from Gefcom 2012. Distinguish between temperature and load prediction.\\
Review papers with related approaches including but not limited to those that can be found under the following link: http://blog.drhongtao.com/2014/08/recommended-papers-for-gefcom2014-contestants.html

\section{Data}
Description of the data made available for the competition
\subsection{Temperature Data}
\subsection{Load Data}
\subsection{Basic exploration with Time Series Analysis Methods}
Autocorrelation, include decompositions?

\section{Feature 'Extraction'}
Description of the features obtained from the data. 
\subsection{Calendar Features}
hour, TOY vs. month

\section{Models}
Description of the Models (LM), GAM (more extensive), NN, RF

\section{Analysis}
What combination of features and models for temperature and load provide us with a good prediction accuracy with respect to Gefcom leaderboard?

\subsection{Error Measures}
Introduce error measures (RMSE, MAE, MAPE, PINBALL) and their differences here? Too late?

\subsection{Temperature Modeling}
\subsubsection{Data Processing}
average temperature vs. principal component

\subsubsection{Effect on Load Prediction}
Effect of temperature on load prediction evaluated using different methods:\\
Mean over past years (yearly lag), LM, GAM, NN, RF vs. true temperature

\subsection{Load Modeling}

\subsubsection{Performance of different Methods}
GAM, NN, RF 

\subsubsection{Model Formulas}
Evaluating the performance of different model formulas built with the features (GAM)


\section{Conclusion}
Draw conclusion based on the analysis done in the main part.

\end{document}